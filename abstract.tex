% vim: set spell spelllang=en tw=100 :

\documentclass[letterpaper]{article}
\usepackage[pass]{geometry}

\usepackage{ijcai16}
\usepackage{times}
\usepackage{microtype}
\usepackage{complexity}

\title{Solving Hard Subgraph Problems in Parallel}

\author{Ciaran McCreesh\thanks{This work was supported by the Engineering and Physical Sciences
Research Council [grant number EP/K503058/1]} \\
University of Glasgow, Glasgow, Scotland \\ c.mccreesh.1@research.gla.ac.uk}

\pdfinfo{
    /Title (Solving Hard Subgraph Problems in Parallel)
    /Author (Ciaran McCreesh)
}

\begin{document}

\maketitle

\begin{abstract}
    We look at problems involving finding subgraphs in larger graphs, such as the maximum clique
    problem, the subgraph isomorphism problem, and the maximum common subgraph problem. We
    investigate variable and value ordering heuristics, different inference strategies, intelligent
    backtracking search (backjumping), and bit- and thread-parallelism to exploit modern hardware.
\end{abstract}

\section{Overview}

We are looking at problems which involve finding subgraphs inside larger graphs. The \emph{maximum
clique} problem  is to find a largest complete subgraph inside a given graph. More generally, the
\emph{subgraph isomorphism} family of problems involve finding a specified pattern graph
inside a larger target graph. More generally still, the \emph{maximum common subgraph}
problem is to find a largest subgraph common to two or more given graphs.
These problems are \NP-hard, but we can often solve instances quickly in practice, even for graphs
with up to tens of thousands of vertices. Algorithms for these problems have been used successfully
in areas including bioinformatics and chemistry, computer vision, law enforcement, model checking,
compilers, and pattern recognition.

Ultimately, all of these problems are solved using some form of backtracking search: we try to build
up an assignment of values to variables, satisfying a set of constraints. To obtain the best
results, we need to select the right kind of preprocessing, inference, learning, and search.

\section{Our Contributions}

When branching, selecting a good variable and value can have a huge impact on search: with good
heuristics, most instances that occur in practice are reasonably easy to solve.

Modern clique algorithms often use some variation of an integrated colouring-based bound and
ordering heuristic \cite{Tomita:2010}. We have determined why this particular branching strategy
works so well in practice \cite{cp/McCreeshP14}. Our results suggest that it is because this
strategy tends to branch on small colour classes first, which is similar to the widely-used smallest
domain first heuristic (and this contradicts the explanation given previously in the literature).
This knowledge allowed us to design a better branching strategy, which explicitly considered the
sizes of colour classes.

For subgraph isomorphism, we observed that many existing benchmark suites contain large numbers of
very easy satisfiable instances. We have shown how to generate ``really hard'' random instances for
subgraph isomorphism problems \cite{ijcai/McCreeshPT15}, and explained how this lets us design
better variable and value ordering heuristics. This work builds upon the satisfiable-unsatisfiable
phase transition phenomena commonly seen in \NP-complete problems, but because of the larger number
of instance generation parameters, much richer behaviour is observed.  Interestingly, our results
suggest that constrainedness \cite{Gent:1996:Kappa}, not proximity to a phase transition, determines
whether an instance is hard in practice.

Since we are looking for good performance, it is important to design algorithms with modern hardware
features in mind. Current CPUs have several (or dozens of) processing cores, and it is wasteful not
to exploit this. We have shown that it is possible to get very good speedups by using
thread-parallel tree search for the maximum clique problem \cite{algorithms/McCreeshP13}. To explain
why our approach worked so well (and why randomised work stealing does not), we then looked in more
detail at how parallel search strategies interact with search.  We show that there is a connection
between parallel tree-search and value ordering heuristic behaviours: by explicitly stealing work
early in search, we can offset poor early value ordering heuristic choices \cite{topc/McCreeshP15},
giving an alternative to discrepancy searches. An additional benefit of this approach is
that it guarantees that adding more processing cores will never make behaviour worse
\cite{Trienekens:1990}, and that parallel runtimes are reproducible.

We brought all of this together to develop a new algorithm for the subgraph isomorphism problem
\cite{cp/McCreeshP15}. We combined bit-parallel filtering, a new all-different propagator,
a stronger form of inference via preprocessing, and a different way of performing backjumping that
does not need to maintain conflict sets or modify propagators. We showed how backjumping may be
parallelised safely and effectively, by treating the branching as a lazy fold with left-zero
elements.

Our results indicated that different instances would benefit from different strengths of filtering. We
demonstrated that it is possible to use using machine learning to select, on an instance-by-instance
basis, a good choice from a portfolio of subgraph isomorphism algorithms \cite{lion/KotthoffMS15}.

We have also investigated generalisations of these problems: the balanced induced biclique problem
\cite{cpaior/McCreeshP14} requires symmetry breaking, and the maximum labelled clique problem is a
two-pass multi-objective problem with additional side constraints \cite{ol/McCreeshP15}.

We are currently investigating the maximum weight clique (or independent set) problem. The colouring
bound used in the maximum clique problem is not usable with large weights. We have seen favourable
initial results using decision diagram bounds \cite{Bergman:2014}, although variable ordering
heuristics behave unexpectedly compared to in conventional tree-search. We also believe that
parallel search is likely to be beneficial here, and that existing work on safe and reproducible
parallel search should generalise to parallel decision diagrams \cite{BergmanCSSSH14}.

We have preliminary results on the maximum common subgraph problem. The best existing
approaches fall into one of two categories: backtracking search with strong inference based around
an enhanced all-different propagator, \cite{NdiayeS11}, or by reduction to the maximum clique
problem. We believe a hybrid approach is possible.

Finally, we intend to investigate stronger conflict analysis for these algorithms, and to move
towards a clause-learning type approach. Combining SAT-style clause learning with graph-based search
has been shown to be effective for colouring problems \cite{Zhou}. We would like to understand why,
and how to generalise this to subgraph isomorphism problems (ideally without having to reduce to
CNF, which loses high level structural information).

\bibliographystyle{named}
\bibliography{abstract}

\end{document}

